\textbf{E1.} (2,5 puntos)

\vspace{20px}
\textit{Solución:}
\\

\begin{enumerate}
[label=\alph*)]
    \item Se puede calcular la energía electrostática en la región del espacio a partir del campo según la expresión:

    \begin{equation*}
        W_e = \frac{1}{2} \int_{\mathcal V} \varepsilon \, E^2 \, dv
    \end{equation*}

    En la región del espacio la permitividad es $\varepsilon_0$. Como el campo es constante en toda la región, se puede sacar de la integral y
    la energía total queda:

    \begin{equation*}
        W_e = \frac{1}{2} \varepsilon_0 E_0^2   \frac{4}{3}   \pi    R^3 =  \frac{2}{3}  \pi  \varepsilon_0 E_0^2 R^3
    \end{equation*}

    \vspace{20px}

    \item Según los resultados del ejemplo 9.4, en el que la esfera conductora está unida a tierra, el potencial en el exterior de dicha esfera es:

    \begin{equation*}
        V(r, \theta) = - E_0 ( 1 - \frac{a^3}{r^3})  r  \cos\theta
    \end{equation*}

    Podemos entonces calcular el campo eléctrico a partir de la relación $\textbf{E} = - \nabla V$.

    El gradiente en coordenadas esféricas viene dado por:

    \begin{equation*}
        \nabla V = \frac{\partial V}{\partial r} \textbf{u}_r +
        \frac{1}{r}\frac{\partial V}{\partial \theta} \textbf{u}_\theta +
        \frac{1}{r \sen\theta}\frac{\partial V}{\partial \varphi} \textbf{u}_{\varphi}
    \end{equation*}

    El campo eléctrico resulta:

    \begin{align*}
        \textbf{E} & = \frac{\partial}{\partial r} \biggl[E_0 \Bigl(1 - \frac{a^3}{r^3} \Bigr) r \cos\theta \biggr] \textbf{u}_r +
        \frac{1}{r}\frac{\partial}{\partial \theta} \biggl[E_0 \Bigl(1 - \frac{a^3}{r^3} \Bigr) r \cos\theta \biggr] \textbf{u}_{\theta} \\[6px]
        & = E_0 \cos\theta \Bigl(1 + \frac{2a^3}{r} \Bigr) \textbf{u}_r -
        E_0  \sen\theta\Bigl(1 - \frac{a^3}{r^3} \Bigr)  \textbf{u}_{\theta}
    \end{align*}

    \vspace{20px}

    \item El sistema del problema es un sistema equivalente al de 9.4, en el que la esfera conductora está conectada a tierra, y por tanto, no puede acumular carga electrostática.
    En ese caso, la energía de la esfera sumergida en el campo es 0.

    \vspace{20px}

    \item Tal y como se ha modelado el problema, no existe un cambio de potencial entre el infinito ($V = 0$), y el potencial que tiene la esfera conductora
    conectada a tierra. Por tanto, el trabajo que cuesta introducir la esfera en la región donde se establece el campo es 0.

\end{enumerate}