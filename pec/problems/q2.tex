\textbf{Q2.} (0,5 puntos)


\vspace{20px}
\textit{Solución:}
\\

La métrica de Coulomb nos dice que la divergencia de $\textbf{A}$ es igual a 0, $\nabla \cdot \textbf{A} = 0$.

En coordenadas cilíndricas la divergencia está dada por la expresión:

\begin{equation*}
    \nabla \cdot \textbf{A} = \frac{1}{\rho} \frac{\partial}{\partial \rho} ( \rho \, A_\rho )
    + \frac{1}{\rho} \frac{\partial}{\partial \varphi} A_{\varphi}
    + \frac{\partial}{\partial z} A_z
\end{equation*}

Comprobamos los potenciales:

\begin{align*}
    \nabla \cdot \textbf{A}_1 & = \frac{\partial}{\partial z} ( C \ln\rho) = 0 \\
    \nabla \cdot \textbf{A}_2 & = \frac{1}{\rho} \frac{\partial}{\partial \rho} \biggl( \rho \; \frac{D}{\rho} \biggr)
    + \frac{\partial}{\partial z} ( C \ln\rho) = 0
\end{align*}

Vemos que ambos cumplen la métrica de Coulomb.

El campo magnético $\textbf{B}$ se calcula a partir del rotacional del potencial vector magnético, $ \textbf{B} = \nabla \times \textbf{A}$.

El rotacional en coordenadas cilíndricas está dado por:

\begin{equation*}
    \nabla \times \textbf{A} =
    \begin{vmatrix}
        \textbf{u}_\rho         & \rho \, \textbf{u}_\varphi & \textbf{u}_z          \\[6px]
        \partial / \partial\rho & \partial / \partial\varphi & \partial / \partial z  \\[6px]
        A_\rho                  & \rho \, A_\varphi          & A_z                    \\[6px]
    \end{vmatrix}
\end{equation*}

Calculamos $\textbf{B}$ a partir de los dos potenciales.

\begin{align*}
    \nabla \cdot \textbf{A}_1 & = \frac{\partial}{\partial \varphi} ( C \ln\rho) \textbf{u}_\rho -
    \rho \, \frac{\partial}{\partial \rho} ( C \ln\rho) \textbf{u}_\varphi  = - C \, \textbf{u}_\varphi  \\
    \nabla \cdot \textbf{A}_2 & = \frac{\partial}{\partial \varphi} ( C \ln\rho) \textbf{u}_\rho +
    \rho \, \frac{\partial}{\partial z} ( \frac{D}{\rho}) \textbf{u}_\varphi  -
    \frac{\partial}{\partial \varphi} (  \frac{D}{\rho}) \textbf{u}_z -
    \rho \, \frac{\partial}{\partial \rho} ( C \ln\rho) \textbf{u}_\varphi  = - C \, \textbf{u}_\varphi
\end{align*}

El resultado es el mismo para los dos potenciales vectores magnéticos, por lo que hemos demostrado que producen el mismo campo magnético,
$\textbf{B} = - C \, \textbf{u}_\varphi$.