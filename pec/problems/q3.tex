\textbf{Q3.} (0,5 puntos)


\vspace{20px}
\textit{Solución:}
\\

Un campo magnético debe ser solenoidal, es decir, las líneas de campo deben ser cerradas, lo que significa que no tienen
ni fuentes ni sumideros.

Por tanto, si $\nabla \cdot \textbf{F} = 0$, $\textbf{F}$ puede ser un campo magnético.

La divergencia en coordenadas esféricas viene dada por:

\begin{equation*}
    \nabla \cdot \textbf{A} = \frac{1}{r^2} \frac{\partial}{\partial r} ( r^2 \, A_r) +
    \frac{1}{r \, \sen\theta} \frac{\partial}{\partial\theta}
    (A_\theta \, \sen\theta )
    +     \frac{1}{r \, \sen\theta} \frac{\partial}{\partial\varphi} A_{\varphi}
\end{equation*}

Calculamos la divergencia de $\textbf{F}$:

\begin{align*}
    \nabla \cdot \textbf{F} & = \frac{1}{r^2} \frac{\partial}{\partial r} ( r\,F_0 \, \cos\theta) +
    \frac{1}{r \, \sen\theta} \frac{\partial}{\partial\theta}
    \biggl(
    - \frac{1}{2}  \frac{F_0}{r} \sen^2{\theta}
    \biggr)
    + \frac{1}{r \, \sen\theta} \frac{\partial}{\partial \varphi}
    \biggl(
    \frac{F_0}{r}
    \biggr) \\
    & = \frac{1}{r^2} \, F_0 \, \cos\theta + \frac{1}{r\, \sen\theta} ( - \frac{F_0}{r} \sen\theta \cos\theta ) = 0
\end{align*}

Por lo tanto, $\textbf{F}$ puede representar un campo magnético $\textbf{B}$.
