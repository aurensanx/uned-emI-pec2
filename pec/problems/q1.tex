\textbf{Q1.} (1 punto)


\vspace{20px}
\textit{Solución:}
\\

El potencial $V(r)$ se nos da en coordenadas esféricas y se nos pide expresarlo en coordenadas cilíndricas. Para ello,
utilizamos las relaciones entre este par de sistemas de coordenadas:

\begin{align*}
    & \rho = r \; \sen\theta \\
    & \varphi = \varphi \\
    & z = r \; \cos\theta
\end{align*}

De estas relaciones obtenemos:

\begin{equation*}
    \rho^2 + z^2 = r^2 \hspace{12px} \Rightarrow  \hspace{12px} r = \sqrt{ \rho^2 + z^2}
\end{equation*}

Podemos entonces expresar el potencial en coordenadas cilíndricas como:

\begin{equation*}
    V(\rho, z) = \frac{A}{\sqrt{ \rho^2 + z^2}}
\end{equation*}

La ecuación de Laplace en coordenadas cilíndricas toma la forma:

\begin{equation*}
    \frac{1}{\rho} \frac{\partial}{\partial \rho} \biggl( \rho \, \frac{V}{\partial \rho} \biggr)
    +
    \frac{1}{\rho^2}  \biggl( \frac{\partial^2 V}{\partial \varphi^2} \biggr)
    +
    \frac{\partial^2 V}{\partial z^2} = 0
\end{equation*}

Calculamos las derivadas parciales:

\begin{align*}
    \frac{\partial V}{\partial \rho}   & = - \frac{A \, \rho}{(\rho^2 + z^2)^{3/2}} & \\[12px]
    \frac{1}{\rho} \frac{\partial}{\partial \rho}  \biggl( \rho \, \frac{V}{\partial \rho} \biggr)  & =
    \frac{A ( \rho^2 - 2z^2)}{(\rho^2 + z^2)^{5/2}} & (1) \\[12px]
    \frac{\partial V}{\partial \rho} &= 0 & (2) \\[12px]
    \frac{\partial V}{\partial z} &= - \frac{A \, z}{(\rho^2 + z^2)^{3/2}} &  \\[12px]
    \frac{\partial^2 V}{\partial z^2} &= \frac{A (- \rho^2 + 2z^2)}{(\rho^2 + z^2)^{5/2}} & (3)
\end{align*}

Sumando (1), (2) y (3) comprobamos que $\Delta^2 V = 0$ y, por tanto, $V$ cumple la ecuación de Laplace en coordenadas cilíndricas.




