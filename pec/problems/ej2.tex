\textbf{E2.} (3 puntos)


\vspace{20px}
\textit{Solución:}
\\

\begin{enumerate}
[label=\alph*)]
    \item Primero, hacemos coincidir el cable de alta tensión con una de las equipotenciales que genera una línea con densidad de carga
    lineal $\lambda$.

    La posición de esta línea de carga está dada por la relación $h^2 = d^2 + a^2$, siendo $h$ la altura del centro del cable de alta tensión, $a$ el radio del
    cable y $d$ la posición de la densidad de carga lineal que genera el cilindro que coincide con el cable de alta tensión.

    Entonces, la distribución de cargas imagen que resuelve el problema es una línea con densidad lineal de carga $-\lambda$ en la posición $(0, -d, 0)$, es
    decir, debajo del suelo una distancia $d$.

    \vspace{20px}

    \item Siendo $d$ la distancia de la posición de la línea de carga ficticia mencionada en el apartado anterior hasta el suelo, la expresión del potencial es
    la que sigue:

    \begin{equation*}
        V = - \frac{\lambda}{2\pi\varepsilon_0} \ln{\frac{|y - d|}{|y+d|}}
    \end{equation*}

    Calculamos el campo eléctrico mediante su relación con el potencial eléctrico:

    \begin{equation*}
        \textbf{E} = \frac{\lambda}{2\pi\varepsilon_0} \frac{(y+d)}{(y-d)} \frac{(2d)}{(y+d)^2} \textbf{u}_y
        = - \frac{\lambda d}{\pi \varepsilon_0 (d^2 - y^2)}  \textbf{u}_y
    \end{equation*}

    Utilizando la relación del potencial con la constante $K$ para equipotenciales de la línea de carga:

    \begin{equation*}
        V = \frac{\lambda}{2\pi\varepsilon_0} \ln{K},
    \end{equation*}

    podemos escribir el campo eléctrico como:

    \begin{equation*}
        \textbf{E} = - \frac{2 V_0 d}{(d^2 - y^2) \ln{K} }  \textbf{u}_y
    \end{equation*}

    \vspace{20px}

    \item Calculamos la constante K de la expresión anterior con los datos proporcionados:

    \begin{equation*}
        K^2 - 2K\frac{h}{a} + 1 = 0 \hspace{12px} \Rightarrow \hspace{12px}  K^2 - 2400K + 1 = 0
    \end{equation*}

    De las 2 soluciones de esta ecuación, nos quedamos con la que hace $K > 1$, es decir, $K \sim 2400$.

    Sustituyendo el resto de datos en la expresión del campo eléctrico descrita en el apartado b, tenemos:

    \begin{equation*}
        \textbf{E} = - \frac{2 \times 4 \times 10^5 V}{ \sqrt {12^2-0,01^2}  m^2  \ln{2400}}  \textbf{u}_y = -8,565 \frac{kV}{m}  \textbf{u}_y
    \end{equation*}

    \vspace{20px}

    \item En el exterior del cable tenemos $y = h - a$. Utilizando la expresión del campo eléctrico calculada anteriormente, tenemos:


    \begin{equation*}
        \textbf{E} = - \frac{V_0 \sqrt{h^2 - a^2}}{ a (h - a) \ln{K}}  \textbf{u}_y
    \end{equation*}

    Evaluando los valores numéricos, obtenemos un campo de $- 5,144 \; MV /m \; \textbf{u}_y$.

    Como este resultado es mayor que la rigidez dieléctrica del aire, que es el valor límite en el cual el aire (o cualquier otro material)
    pierde su capacidad aislante y pasa a ser conductor, se produce la ruptura dieléctrica y se producirá una chispa y una descarga.

\end{enumerate}