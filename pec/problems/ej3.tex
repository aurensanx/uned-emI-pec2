\textbf{E3.} (2,5 puntos)


\vspace{20px}
\textit{Solución:}
\\

\begin{enumerate}
[label=\alph*)]
    \item Que el condensador sea neutro globalmente quiere decir que su carga total es cero. Eso solo puede suceder si la suma de la densidad de carga superficial
    $\sigma_0$ en la superficie $S$ más la densidad uniforme de carga $\rho_0$ en el volumen del medio es cero. Por lo tanto:

    \begin{equation*}
        S \, \sigma_0 + \rho_0 \, S \, l = 0 \hspace{12px} \Rightarrow \hspace{12px} \sigma_0 = - l \, \rho_0
    \end{equation*}


    \vspace{20px}

    \item Por la simetría del problema, todos los vectores tendrán la misma dirección, perpendicular a las placas. Fuera del condensador los vectores son nulos.

    En el instante inicial, el vector desplazamiento $\textbf{D}$ presenta una discontinuidad debido a la densidad de carga superficial $\sigma_0$ aplicada.

    Aplicando las ecuaciones constitutivas $\textbf{D} = \varepsilon \textbf{E}$ y $\textbf{J} = \gamma \textbf{E}$ obtenemos la expresión de los campos en el momento
    en el que el medio empieza a ser conductor.

    \begin{align*}
        &\textbf{D}  = \sigma_0 \\[6px]
        & \textbf{E} = \frac{\sigma_0}{\varepsilon} \\[6px]
        & \textbf{J}  = \frac{\gamma}{\varepsilon} \sigma_0
    \end{align*}

    La energía electrostática en ese momento se puede calcular como la integral del campo en el volumen del condensador.

    \begin{equation*}
        W_e = \int_{\mathcal{V}}\varepsilon E^2 dv =  \frac{\sigma_0^2 \, l \, S}{2 \varepsilon}
    \end{equation*}

    \vspace{20px}

    \item Para obtener la expresión que proporciona la evolución temporal de los campos una vez que el medio se hace conductor aplicamos la
    ecuación de continuidad para un volumen que incluya la placa con la densidad de carga $\sigma_0$.

    La expresión que obtenemos es:

    \begin{equation*}
        J (t) =  \frac{\gamma}{\varepsilon} \sigma(t) = - \frac{d \sigma (t)}{dt}
    \end{equation*}

    Esta es una ecuación diferencial de primer grado cuya solución es:

    \begin{equation*}
        \sigma(t) = \sigma_0 \, e^{-\frac{\gamma}{\varepsilon}t}
    \end{equation*}

    Los tres campos tienen la misma expresión funcional:

    \begin{align*}
        & D(t)  = \sigma_0 \, e^{-\frac{\gamma}{\varepsilon}t} \\[6px]
        & E(t) = \frac{\sigma_0}{\varepsilon} \, e^{-\frac{\gamma}{\varepsilon}t} \\[6px]
        & J(t)  = \frac{\gamma}{\varepsilon} \sigma_0 \, e^{-\frac{\gamma}{\varepsilon}t}
    \end{align*}

    \vspace{20px}

    \item La energía disipada por efecto Joule se puede calcular integrando a lo largo de todo el tiempo la potencia disipada, que
    se puede escribir como función de los campos calculados.

    \begin{align*}
        W & =  \int_0^{\infty} P \, dt =  \int_0^{\infty} \int_{\mathcal{V}} \gamma \, E^2 \, dv \, dt =
        \int_0^{\infty} \int_{\mathcal{V}} \gamma \, \frac{\sigma_0^2}{\varepsilon^2}  \, e^{-2\frac{\gamma}{\varepsilon}t}  \, dv \, dt = \\[6px]
        & =  \gamma \, \frac{\sigma_0^2}{\varepsilon^2}  \, l \, S  \int_0^{\infty} e^{-2\frac{\gamma}{\varepsilon}t} \, dt =
        \frac{\sigma_0^2 \, l \, S}{2 \varepsilon}
    \end{align*}

    Este es el mimo valor obtenido en el apartado b para la energía electrostática inicial, por lo que toda la energía del sistema se disipa
    en el momento que toda la carga queda neutralizada, lo que sucede cuando el tiempo tiende a infinito.

\end{enumerate}